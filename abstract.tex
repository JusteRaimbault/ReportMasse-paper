\documentclass[3p,times,procedia]{elsarticle}


\flushbottom

%% The `ecrc' package must be called to make the CRC functionality available
\usepackage{ecrc}
%\usepackage{amsmath}


%% The ecrc package defines commands needed for running heads and logos.
%% For running heads, you can set the journal name, the volume, the starting page and the authors

%% set the volume if you know. Otherwise `00'
\volume{00}

%% set the starting page if not 1
\firstpage{1}

%% Give the name of the journal
\journalname{24th EURO Working Group on Transportation Meeting, EWGT 2021, 8-10 September 2021, Aveiro, Portugal}

%% Give the author list to appear in the running head
%% Example \runauth{C.V. Radhakrishnan et al.}
\runauth{A. Nacer-Weill, C. Yang, T. Barbet and J Raimbault}

%% The choice of journal logo is determined by the \jid and \jnltitlelogo commands.
%% A user-supplied logo with the name <\jid>logo.pdf will be inserted if present.
%% e.g. if \jid{yspmi} the system will look for a file yspmilogo.pdf
%% Otherwise the content of \jnltitlelogo will be set between horizontal lines as a default logo

%% Give the abbreviation of the Journal.
\jid{trpro}

%% Give a short journal name for the dummy logo (if needed)
%\jnltitlelogo{Transportation Research}

%% Hereafter the template follows `elsarticle'.
%% For more details see the existing template files elsarticle-template-harv.tex and elsarticle-template-num.tex.

%% Elsevier CRC generally uses a numbered reference style
%% For this, the conventions of elsarticle-template-num.tex should be followed (included below)
%% If using BibTeX, use the style file elsarticle-num.bst

%% End of ecrc-specific commands
%%%%%%%%%%%%%%%%%%%%%%%%%%%%%%%%%%%%%%%%%%%%%%%%%%%%%%%%%%%%%%%%%%%%%%%%%%

%% The amssymb package provides various useful mathematical symbols

\usepackage{amssymb}
%% The amsthm package provides extended theorem environments
%% \usepackage{amsthm}

%% The lineno packages adds line numbers. Start line numbering with
%% \begin{linenumbers}, end it with \end{linenumbers}. Or switch it on
%% for the whole article with \linenumbers after \end{frontmatter}.
%% \usepackage{lineno}

%% natbib.sty is loaded by default. However, natbib options can be
%% provided with \biboptions{...} command. Following options are
%% valid:

%%   round  -  round parentheses are used (default)
%%   square -  square brackets are used   [option]
%%   curly  -  curly braces are used      {option}
%%   angle  -  angle brackets are used    <option>
%%   semicolon  -  multiple citations separated by semi-colon
%%   colon  - same as semicolon, an earlier confusion
%%   comma  -  separated by comma
%%   numbers-  selects numerical citations
%%   super  -  numerical citations as superscripts
%%   sort   -  sorts multiple citations according to order in ref. list
%%   sort&compress   -  like sort, but also compresses numerical citations
%%   compress - compresses without sorting
%%
\biboptions{authoryear}

% \biboptions{}

% if you have landscape tables
\usepackage[figuresright]{rotating}
%\usepackage{harvard}
% put your own definitions here:x
%   \newcommand{\cZ}{\cal{Z}}
%   \newtheorem{def}{Definition}[section]
%   ...

% add words to TeX's hyphenation exception list
%\hyphenation{author another created financial paper re-commend-ed Post-Script}

% declarations for front matter

\usepackage[bookmarks=false]{hyperref}
    \hypersetup{colorlinks,
      linkcolor=blue,
      citecolor=blue,
      urlcolor=blue}

\usepackage{multicol}




\begin{document}
\begin{frontmatter}

%% Title, authors and addresses

\title{An agent-based model for modal shift in public transport}

\author[a]{Thibaut Barbet}
\author[a]{Amine Nacer-Weill}
\author[a]{Changtao Yang}
\author[b]{Juste Raimbault\corref{cor1}}

\address[a]{Ecole des Ponts ParisTech, Champs-sur-Marne, France}
\address[b]{CASA, University College London, London, United Kingdom}

\begin{abstract}

\end{abstract}

\begin{keyword}
Modal shift \sep Public transport disruption \sep Agent-based modeling \sep Model exploration
\end{keyword}
\cortext[cor1]{Corresponding author.}
\end{frontmatter}

\email{j.raimbault@ucl.ac.uk}

%%
%% Start line numbering here if you want
%%
% \linenumbers

%% main text

%\enlargethispage{-7mm}
\section{Introduction}


%Dans les transports en communs aussi les usagers doivent faire des choix en un temps limité et avec des informations limitées. Ils subissent régulièrement des perturbations. Il s’enclenche alors un processus de décision complexe qui va aboutir à des comportements complexes. Le travail de Xavier Brisbois1 notamment a contribué à montrer que l'analyse choix de report modal à long terme ne peut se dispenser d'une analyse des comportements, de la dimension cognitive du choix en complément de variables socio-économiques qui n’expliquentpas nécessairement les comportements.
%En s’inspirant des modèles d’information voyageur comme ceux de Leng et Corman2 qui explorent les effets de l’information voyageur sur les temps de trajet, nous nous posons la question de la complémentarité de ces modélisations multiagents avec des enquêtes de préférences déclarées. 

Disruptions in public transport networks are not rare events, often worsened by an increased complexity of these networks and their management \citep{dekker2018next}. Network resilience is then tightly linked to patterns of modal shift \cite{stamos2015impact}, and a better understanding of these is crucial both from a theoretical and operational viewpoint. In that context, users have to make decisions in a limited time and under partial information. The study of modal shift under disruption is therefore improved when taking into account users behavior and a detailed representation of users cognition \citep{brisbois2010processus}. More particularly, the role of information provided in real time can in some cases become crucial, as rerouting may in fact increase the congestion in other parts of the network and ultimately increase the average travel time \citep{chatterjee2002driver,chorus2006travel}.

To study mechanisms linking information given to users with modal shift, and more generally the evolution of multi-modal network flows under disruptions, agent-based modeling has been highlighted as a relevant approach \citep{leng2020role}. For example, \cite{leng2020issue} show that the issue time of information has a significant impact on the total congestion. Agent-based models are used in similar studies of modal share, such as by \cite{baindur2011agent} for freight. \cite{raney2003agent} describe an application of the Matsim model, which is a data-driven agent-based and activity-based transport model, to a large sample of Switzerland transport network. The Matsim model can be applied at large scales in a reproducible manner, such as in the case of Ile-de-France illustrated by \cite{horl2020reproducible}.

Large transport agent-based models such as Matsim however require an extensive parametrisation on real data, are difficult to systematically validate given their runtime and large parameter space, and despite their high modularity can be tuned only to some extent regarding a precise description of users behavior in public transport and their interactions with a network disruption. This contribution therefore proposes to 




\section{Declared preferences survey}


\section{Model description}

\cite{martin2016strategies} % heterogeneity in user behavior

\section{Model exploration}


\section{Discussion}


%% References
%%
%% Following citation commands can be used in the body text:
%% Usage of \cite is as follows:
%%   \cite{key}         ==>>  [#]
%%   \cite[chap. 2]{key} ==>> [#, chap. 2]
%%

%The citation must be used in following style: \cite{article-minimal} \cite{article-full} \cite{article-crossref} \cite{whole-journal}.
%% References with BibTeX database:

%
\bibliographystyle{elsarticle-harv}
\bibliography{biblio.bib}

\clearpage


\end{document}

%%%%
% Template

%
%\begin{figure}[t]\vspace*{4pt}
%%\centerline{\includegraphics{fx1}\hspace*{5mm}\includegraphics{fx1}}
%\centerline{\includegraphics{gr1}}
%\caption{(a) first picture; (b) second picture.}
%\end{figure}
%
%\begin{table}[h]
%\caption{An example of a table.}
%\begin{tabular*}{\hsize}{@{\extracolsep{\fill}}lll@{}}
%\toprule
%An example of a column heading & Column A ({\it{t}}) & Column B ({\it{t}})\\
%\colrule
%And an entry &   1 &  2\\
%And another entry  & 3 &  4\\
%And another entry &  5 &  6\\
%\botrule
%\end{tabular*}
%\end{table}
%



%
%\begin{equation}
%\begin{array}{lcl}
%\displaystyle X_r &=& \displaystyle\dot{Q}^{''}_{rad}\left/\left(\dot{Q}^{''}_{rad} + \dot{Q}^{''}_{conv}\right)\right.\\[6pt]
%\displaystyle \rho &=& \displaystyle\frac{\vec{E}}{J_c(T={\rm const.})\cdot\left(P\cdot\left(\displaystyle\frac{\vec{E}}{E_c}\right)^m+(1-P)\right)}
%\end{array}
%\end{equation}
%
%



%% The Appendices part is started with the command \appendix;
%% appendix sections are then done as normal sections
%% \appendix

%% \section{}
%% \label{}
%
%\appendix
%\section{An example appendix}
%Authors including an appendix section should do so before References section. Multiple appendices should all have headings in the style used above. They will automatically be ordered A, B, C etc.
%
%\subsection{Example of a sub-heading within an appendix}
%There is also the option to include a subheading within the Appendix if you wish.
%



%
%
%\section*{Acknowledgements}
%
%Acknowledgements and Reference heading should be left justified, bold, with the first letter capitalized but have no numbers. Text below continues as normal.
%

